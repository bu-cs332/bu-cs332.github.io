\documentclass[letterpaper,11pt]{article}

\usepackage{fullpage,amsmath,hyperref,color,clrscode,enumitem}

% Unicode compatibility
\usepackage{iftex}
\ifPDFTeX
\usepackage[utf8]{inputenc}
\usepackage[noTeX]{mmap}
\usepackage[T1]{fontenc}
\fi
% XeTeX does not support mmap
\ifLuaTeX
\usepackage{luatex85}
\usepackage[noTeX]{mmap}
\fi

\addtolength{\textwidth}{0.2in}
\addtolength{\oddsidemargin}{-0.1in}
\addtolength{\evensidemargin}{-0.1in}

\addtolength{\textheight}{0.5in}
\addtolength{\topmargin}{-0.25in}

\begin{document}
{\noindent\large
CS 332: {\em Theory of Computation} \hfill Prof. Ran Canetti\\
Boston University \hfill \today\\}
\vspace{1pt} \hrulefill\vspace{3mm}
\begin{center}
{\Large\bf Homework 1 -- Due Tuesday, September 15, 2020 \underline{before} 2:00PM}
\end{center}

\paragraph{Reminder}
%\begin{itemize}
%\item
Collaboration is permitted, but you must write the solutions {\em by
yourself without assistance}, and be ready to explain them orally to
the course staff if asked. You must also identify your
collaborators. Getting solutions from outside sources such as the
Web or students not enrolled in the class is strictly forbidden.
%\end{itemize}

%\vspace*{-2mm}


\paragraph{Exercises}
Please practice on exercises and solved problems in Chapters 1.
The material they cover may appear on exams. It's also a good idea to practice constructing DFAs and NFAs for specific languages using \url{http://automatatutor.com/}.

\paragraph{Problems}
There are \ref{last}
mandatory problems and one bonus problem.


\begin{enumerate}
\item
	Give state diagrams of DFAs with as few states as you can recognizing the following languages.
	\begin{enumerate}
    \item[(a)] The alphabet is the set of Nucleic acids $\Sigma = \{A, C, G, T\}$. \\
    \(L_1 = \{w\mid w \text{ is a sequence of correct  pairings of Nucleic acids}\}\). (Recall that the correct pairings are  $\{A,T\}$ and $\{C,G\}$). In other words,  a string is in $L_1$ if the first and second acids  are a corect pairing, the third and fourth acids are a correct pairing,  and so on. %DFA
		
    \item[(b)] $\Sigma = \{b, e, o\}$. \\
    $L_2 = \{w \mid w \text{ contains the word ``oboe'' as a substring}\}$.
    
		\item[(c)] $\Sigma = \{0, 1\}$. \\
    \(L_3 = \{w\mid w \text{ represents a \emph{valid} binary number in big endian that is a multiple of }3\}\). In other words, and the number is presented starting from the most significant bit and \emph{cannot} have leading 0s.
    For example, 0, 11, 110, 1001 are in $L_3$, but 00, 1, 011 are not.
    \\
    \(L_4 = \{w\mid w \text{ represents a binary number in little endian that is congruent to 1 modulo 3}\}\). In other words, the number equals $3n + 1$ for some integer $n \ge 0$, and the number is presented starting from the least significant bit and \emph{can} have trailing 0s. For example, 1, 0010, 111 are in $L_4$.
     %DFA
	\end{enumerate}
	Give state diagrams of NFAs with as few states as you can recognizing the following languages:
	\begin{enumerate}
		\item[(d)] $\Sigma = \{1\}$. \\
    \(L_5 =\{w \mid \text{the length of }w\text{ is divisible by 3 or by 5}\}\). %NFA
		
		\item[(e)] $\Sigma = \{a, d\}$. \\
    \(L_6 = \{w \mid w \text{ contains substrings ``add'' and ``dad'' which do not overlap}\}\). %NFA
	\end{enumerate}
  You can optionally provide some reasoning for your DFA/NFA, which could be accounted for partial credits if the construction is not completely correct.

\item\label{last}
  \begin{enumerate}
    \item For any finite set $\Sigma$ and finite set $L \subseteq \Sigma^*$, prove that $L$ is regular.
    \item Let $\Sigma = \{0, 1\}$. Prove that for any $k > 0$, the language $L_k := \{ w \in \Sigma^* \mid w\text{ ends with } 0^k\}$ can be recognized by an NFA, where $0^k$ means the 0 character repeated $k$ times. (for this part, give formal descriptions for the NFA you construct)
  \end{enumerate}

\item
	\textbf{Bonus problem, no collaboration is allowed}
	
  Given a regular language $L$.
  \begin{enumerate}
    \item Let $f_k(L) := \{a \mid \exists b : ab \in L\text{ and } |b| = k\}$. Prove that for every $k > 0$, $f_k(L)$ is regular. \\
      \emph{Hint}: Start with $f_1$.
    \item Let $double(L) := \{a \mid \exists b : ab \in L\text{ and } |b| = |a|\}$. Prove that $double(L)$ is regular. \\
      \emph{Hint}: Notice that counterintuitively the language $f_k(L)$ does not become ``harder'' when $k$ goes to infinity.
    \item Let $exp(L) := \{a \mid \exists b : ab \in L\text{ and } |b| = 2^{|a|}\}$. Prove that $exp(L)$ is regular.
  \end{enumerate}
\end{enumerate}

\end{document}
